% Класс документа с параметрами: кегль шрифта, шрифт с засечками для формул,
% соотношение сторон слайда
\documentclass[14pt,mathserif,aspectratio=43,unicode]{beamer}
%\setbeameroption{show notes} % for development
\setbeameroption{show only notes} % for materials

%%% Кодировки и шрифты %%%
\usepackage{cmap}				% Улучшенный поиск русских слов в полученном pdf-файле
\usepackage[T2A,T1]{fontenc}			% Поддержка русских букв
\usepackage[utf8]{inputenc}				% Кодировка utf8
\usepackage[english, russian]{babel}	% Языки: русский, английский

\usepackage{xcolor}
% Цвет логотипа лаборатории
\definecolor{lab-blue}{RGB}{30,97,182}
% Ищется в яндексе по запросу «травяной цвет»
\definecolor{grass}{RGB}{93,161,48}

% Математические символы
\usepackage{amsthm,amsfonts,amsmath,amssymb,amscd} % Математические дополнения от AMS

% Перечёркивание текста
\usepackage{ulem}

% Графики
\usepackage{pgfplots}
\pgfplotsset{width=9cm,compat=1.9}
\usepackage{epstopdf}

%%% Поля и разметка страницы %%%
\usepackage{lscape}		% Для включения альбомных страниц
\usepackage{geometry}	% Для последующего задания полей

%%% Оформление абзацев %%%
\usepackage{indentfirst} % Красная строка

%%% Общее форматирование
\usepackage[singlelinecheck=off,center]{caption}	% Многострочные подписи
\usepackage{soul}									% Поддержка переносоустойчивых подчёркиваний и зачёркиваний
 
%%% Изображения %%%
\usepackage{graphicx} % Подключаем пакет работы с графикой
\graphicspath{{images/}} % Пути к изображениям
\usepackage{xmpmulti}

%%% Заголовки и футеры %%%
\usepackage{fancyhdr}

%%% Код
\usepackage{listings}
\lstset{language=Python,
        keywordstyle=\color{grass}\ttfamily,
        stringstyle=\color{red}\ttfamily,
        commentstyle=\color{green}\ttfamily,
        morecomment=[l][\color{magenta}]{\#},
        numbers=left, 
        numberstyle=\color{lightgray}\tiny, 
        numbersep=5pt,
        basicstyle=\footnotesize
}

%%% Filter warnings
\usepackage{silence,lmodern}
\WarningFilter{biblatex}{Patching footnotes failed}

%%% Библиография
\usepackage{hyperref}
\usepackage{csquotes}
\usepackage[backend=biber,
            style=gost-footnote-min,
            sorting=none,
            hyperref=true,
            url=false]{biblatex}
\addbibresource{bibliography.bib}
%\AtEveryCitekey{\iffootnote{\clearfield{url}}{}}

%%% Кодировки и шрифты %%%
\renewcommand{\rmdefault}{ftm} % Включаем Times New Roman

%%% Выравнивание и переносы %%%
\sloppy					% Избавляемся от переполнений
\clubpenalty=10000		% Запрещаем разрыв страницы после первой строки абзаца
\widowpenalty=10000		% Запрещаем разрыв страницы после последней строки абзаца

% Основной цвета beamer'а (по умолчанию — синий)
\setbeamercolor{structure}{fg=black}

% Нумерация слайдов
\setbeamertemplate{footline}{%
\raisebox{7pt}{\makebox[\paperwidth]{%
\hfill\makebox[10pt]{\scriptsize\textcolor{gray}{\insertframenumber~~~~}}}}}

% Начинать нумерацию слайдов не с титульного слайда
% \let\otp\titlepage
% \renewcommand{\titlepage}{\otp\addtocounter{framenumber}{-1}}

% Сноска без номера
%\newcommand\articlenote[1]{%
%  \begingroup%
%  \renewcommand\thefootnote{}\footnote{#1}%
%  \addtocounter{footnote}{-1}%
%  \endgroup%
%}

% Убрать навигацию beamer'а по умолчанию
\beamertemplatenavigationsymbolsempty

% Сноска без номера
\newcommand\articlenote[1]{%
  \begingroup%
  \renewcommand\thefootnote{}\footnote{#1}%
  \addtocounter{footnote}{-1}%
  \endgroup%
}
 
% Параметры для создания титульной страницы
\title{Нейросетевые методы поиска похожих людей в видео}
\author{Лютов Владимир Сергеевич}

\institute{
    \small{ВМК МГУ}

    \smallskip

    \includegraphics[scale=0.6]{lab_logo.eps}
}

% Время
\date{\small{19 декабря 2018}}

\logo{\includegraphics[height=1cm]{lab_logo.eps}\vspace{220pt}}

\newcommand{\w}{\mathbf{w}}
\newcommand{\T}{\mathbf{t}}
\newcommand{\x}{\mathbf{x}}
\newcommand{\pHi}{\mathbf{\phi}}
\newcommand{\Eps}{\mathbf{\epsilon}}
\newcommand{\y}{\mathbf{y}}

\begin{document}


\begin{frame}[plain]
    \titlepage
\end{frame}

\section{О задаче}

\begin{frame}{Классификация спецодежды}

    \begin{center}
        \includegraphics[width=\textwidth]{spec.eps}
    \end{center}
    
\end{frame}

\note[itemize]{
    \item Необходимо: безопасность на предприятиях, прибыль в магазинах
    \item Актуальность: требуется много данных для каждой задачи
}

\begin{frame}{Универсальный дескриптор}

    \begin{center}
        \includegraphics[width=\textwidth]{algo.eps}
    \end{center}
    
\end{frame}

\begin{frame}{Цели и задачи}

    Цель: усовершенствование нейросетевых методов поиска похожих людей в видео.
    
    Задачи:

    \begin{itemize}
        \item Построить и исследовать универсальный алгоритм классификации изображений людей в спецодежде.
        \item Адаптировать алгоритм для работы с изображениями людей, частично перекрытых предметами.
        \item Исследовать алгоритмы определения степени уверенности классификации для задачи классификации спецодежды.
    \end{itemize}
    
\end{frame}

\begin{frame}{Эталонные коллекции}

    \small

    \begin{tabular}{ p{2.5cm} | p{1.5cm} p{1.5cm} p{1.5cm} p{1.5cm} } 
        Название & Кол-во человек & Кол-во кадров & Кол-во камер & Разрешение \\
        \hline
        CUHK03 & 1360 & 131647 & 2 & ~100x300 \\
        Market-1501 & 1501 & 32668 & 6 & 128x64 \\
        DukeMTMC-ReID & 1812 & 38922 & 8 & ~75x200 \\
        Union & 4673 & 84754 & 16 & vary \\
        Detected & 4673 & 84754 & 16 & vary \\
        Mars & 1261 & 1191003 & 6 & 128x256
    \end{tabular}
    
    Для классификации составлена коллекция из 6821 изображений и 4 классов (по мотивам конкурса на Kaggle).
    
\end{frame}

\note[itemize]{
    \item В рамках этой работы из коллекций CUHK03, Market-1501 и DukeMTMC-ReID составлена объединенная коллекция Union.
    \item Так как в этих коллекциях применяются разные алгоритмы обнаружения людей, все изображения людей были дополнительно обработаны единым алгоритмом обнаружения людей.
    \item Для оценки качества классификации была составлена отдельная коллекция из 6821 изображений.
}

\begin{frame}{Работа c Mars}

    \small

    \begin{center}
        \includegraphics[width=0.5\textwidth]{cmc.png}
        \includegraphics[width=0.5\textwidth]{roc.png}
    \end{center}
    
\end{frame}

\note[itemize]{
    \item 100K * 400K * 1000
}

\begin{frame}{Эксперименты с AM-Softmax}

    \begin{center}
        \includegraphics[width=0.5\textwidth]{am-cmc.png}
        \includegraphics[width=0.5\textwidth]{am-roc.png}
    \end{center}
    
\end{frame}

\begin{frame}{Выводы}

    \begin{itemize}
        \item Составлена коллекция для определения спецодежды
        \item Подготовлен код для работы с датасетом Mars
        \item Улучшен алгоритм построения дескриптора
    \end{itemize}
    
\end{frame}

\note[itemize]{
    \item Время заняться нужными экспериментами
}

\end{document}