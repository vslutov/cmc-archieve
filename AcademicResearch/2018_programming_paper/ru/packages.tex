%%% Поля и разметка страницы %%%
\usepackage{lscape}		% Для включения альбомных страниц
\usepackage{geometry}	% Для последующего задания полей

%%% Кодировки и шрифты %%%
\usepackage{cmap}						% Улучшенный поиск русских слов в полученном pdf-файле
\usepackage[T2A]{fontenc}				% Поддержка русских букв
\usepackage{fontenc}				% Поддержка русских букв
\usepackage[utf8]{inputenc}				% Кодировка utf8
\usepackage[english, russian]{babel}	% Языки: русский, английский
%\usepackage{pscyr}						% Красивые русские шрифты

%%% Математические пакеты %%%
\usepackage{amsthm,amsfonts,amsmath,amssymb,amscd} % Математические дополнения от AMS

%%% Оформление абзацев %%%
\usepackage{indentfirst} % Красная строка

%%% Общее форматирование
\usepackage[singlelinecheck=off,center]{caption}	% Многострочные подписи
\usepackage{soul}									% Поддержка переносоустойчивых подчёркиваний и зачёркиваний

%%% Библиография %%%
\usepackage{cite} % Красивые ссылки на литературу
\usepackage[nottoc]{tocbibind}
\makeatletter
\bibliographystyle{utf8gost705u}	% Оформляем библиографию в соответствии с ГОСТ 7.0.5
\renewcommand{\@biblabel}[1]{#1.}	% Заменяем библиографию с квадратных скобок на точку:
\makeatother

%%% Гиперссылки %%%
\usepackage[linktocpage=true,plainpages=false,pdfpagelabels=false]{hyperref}

%%% Изображения %%%
\usepackage{graphicx} % Подключаем пакет работы с графикой
\graphicspath{{images/}} % Пути к изображениям

%%% Заголовки и футеры %%%
\usepackage{fancyhdr}

%%% Графики %%%
\usepackage{pgfplots}
\usepackage{epstopdf}

%%% Кодировки и шрифты %%%
\renewcommand{\rmdefault}{ftm} % Включаем Times New Roman

%%% Выравнивание и переносы %%%
\sloppy					% Избавляемся от переполнений
\clubpenalty=10000		% Запрещаем разрыв страницы после первой строки абзаца
\widowpenalty=10000		% Запрещаем разрыв страницы после последней строки абзаца