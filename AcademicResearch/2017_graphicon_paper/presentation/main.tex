% Класс документа с параметрами: кегль шрифта, шрифт с засечками для формул,
% соотношение сторон слайда
\documentclass[14pt,mathserif,aspectratio=43]{beamer}

% Кодировка файла
\usepackage[utf8]{inputenc}
% Переносы для русского языка
\usepackage[russian]{babel}

\usepackage{xcolor}
% Цвет логотипа лаборатории
\definecolor{lab-blue}{RGB}{30,97,182}
% Ищется в яндексе по запросу «травяной цвет»
\definecolor{grass}{RGB}{93,161,48}

\usepackage[T1]{fontenc}
% Шрифт для формул
\usepackage[bitstream-charter]{mathdesign}

% Математические символы
\usepackage{amsmath}

% Перечёркивание текста
\usepackage{ulem}

% Графики
\usepackage{pgfplots}
\pgfplotsset{width=9cm,compat=1.9}

\usepackage{epstopdf}

% Основной цвета beamer'а (по умолчанию — синий)
\setbeamercolor{structure}{fg=black}

% Путь к картинкам
\graphicspath{{images/}}

% Нумерация слайдов
\setbeamertemplate{footline}{%
\raisebox{7pt}{\makebox[\paperwidth]{%
\hfill\makebox[10pt]{\scriptsize\textcolor{gray}{\insertframenumber~~~~}}}}}

% Начинать нумерацию слайдов не с титульного слайда
% \let\otp\titlepage
% \renewcommand{\titlepage}{\otp\addtocounter{framenumber}{-1}}

% Сноска без номера
\newcommand\articlenote[1]{%
  \begingroup%
  \renewcommand\thefootnote{}\footnote{#1}%
  \addtocounter{footnote}{-1}%
  \endgroup%
}

% Убрать навигацию beamer'а по умолчанию
\beamertemplatenavigationsymbolsempty

% Параметры для создания титульной страницы
\title{\normalsize{Глубинные двоичные дескрипторы изображения человека для его повторной идентификации и сопровождения в видео}}
\author{\small{Лютов Владимир Сергеевич$^1$, \\Конушин Антон Сергеевич$^{1,2}$}}

\institute{
    \small{$^1$МГУ, $^2$ВШЭ}

    \smallskip

    \includegraphics[scale=0.6]{lab_logo.eps}
}

% Время
\date{\small{26 сентября 2017 г.}}

\logo{\includegraphics[height=1cm]{lab_logo.eps}\vspace{220pt}}

\begin{document}

%-------------------------------------------------------------------------------
\begin{frame}[plain]
    \titlepage
\end{frame}

\section{Основное}

%-------------------------------------------------------------------------------
\begin{frame}{Повторная идентификация человека}

    Требуется сопоставить обнаружения людей, в том числе на разных камерах

    % Вертикальный пробел. Есть ещё \smallskip и \bigskip
    \bigskip

    \includegraphics[width=0.5\textwidth]{image1.png}
    \includegraphics[width=0.5\textwidth]{image4.png}
\end{frame}

%-------------------------------------------------------------------------------
\begin{frame}{Дескриптор изображения человека}

    \textcolor{lab-blue}{Повторная идентификация состоит из}
    
    \begin{itemize}
        \item Построение дескриптора изображения человека
        \item Поиск по базе дескрипторов
    \end{itemize}
    
    % Вертикальный пробел. Есть ещё \smallskip и \bigskip
    \bigskip
    
    \begin{center}
        \includegraphics[width=\textwidth]{lutov-image2.eps}
    \end{center}
\end{frame}

%-------------------------------------------------------------------------------
\begin{frame}[label=reason]{Для чего нужно решать задачу}

    \begin{itemize}
        \item Сценарий видеонаблюдения за объектом
        \item Поиск человека (преступника или пропавшего без вести) по записям с видеокамер
        \item Подсчет людей на мероприятиях
    \end{itemize}
    
    \begin{center}
        \includegraphics[width=0.7\textwidth]{image10.png}
    \end{center}
    
    \hyperlink{hard_things}{\beamerbutton{Трудности}}

\end{frame}

%-------------------------------------------------------------------------------
\begin{frame}{Формальная постановка}

    \begin{itemize}
        \item Вход: изображение человека $I \in \mathbb{R}^{N \times M \times 3}$.
        \item Выход: вектор-дескриптор $\nu = A(I) \in \mathbb{R}^K$
        \item $\max \mathbb{P}(\rho(A(I_1), A(I_2)) < \rho(A(J_1), A(J_2)))$, где \\
        $I_1, I_2$ --- изображения одного человека\\
        $J_1, J_2$ --- изображения разных людей\\
        $\rho(\cdot, \cdot)$ --- выбранная метрика, например, $L_2$
    \end{itemize}
    
    \begin{center}
        \includegraphics[width=0.8\textwidth]{image5.eps}
    \end{center}
    
\end{frame}

%-------------------------------------------------------------------------------
\begin{frame}{Цель работы}

    Усовершенствование алгоритмов построения дескрипторов изображения человека для повышения скорости работы алгоритма
    
    \begin{center}
        \includegraphics{image6.eps}
    \end{center}
    
\end{frame}

%-------------------------------------------------------------------------------
\begin{frame}{Обзор существующих подходов}

    \textcolor{lab-blue}{Методы без машинного обучения}

    \begin{itemize} \footnotesize 
        \item[] Показывают недостаточную точность, по сравнению с аналогами [Srikrishna, 2016]
    \end{itemize}
    
    \textcolor{lab-blue}{Методы с машинным неглубинным обучением}
    
    \begin{itemize} \footnotesize 
        \item[] Наилучший -- иерархический гауссовый дескриптор [Matsukawa, 2016]
    \end{itemize}
    
    \textcolor{lab-blue}{Методы с глубинным обучением}
    
    \begin{itemize} \footnotesize 
        \item[] Наилучшие -- модификации современных классифицирующих нейросетей [Hermans, 2017][Zhang,2017]
        \item[] Предложенный метод -- модификация VGG16
    \end{itemize}
    
\end{frame}

%-------------------------------------------------------------------------------
\begin{frame}{Предложенный алгоритм}
    
    \begin{center}
        \includegraphics[height=0.85\textheight]{image3.eps}
    \end{center}
    
\end{frame}

%-------------------------------------------------------------------------------
\begin{frame}[label=market1501_benchmark]{Оценка вещественных дескрипторов}

    \begin{tikzpicture}
    \begin{axis}[
        ybar, axis on top,
        font=\small,
        height=0.7\textheight,width=\textwidth,
        bar width=0.7cm,
        ymajorgrids,
        major grid style={draw=white},
        enlarge y limits={value=.1,upper},
        ymin=0, ymax=100, ytick={0,20,40,60,80,100},
        axis x line*=bottom,
        y axis line style={opacity=0},
        tickwidth=0pt,
        enlarge x limits=true,
        legend style={
            at={(0.5,-0.2)},
            anchor=north,
            legend columns=1,
            /tikz/every even column/.append style={column sep=0.5cm}
        },
        ylabel={rank-1, \%},
        symbolic x coords={
            GOG,
            MobileNet+DML,
            ResNet+TL,
            Предл. метод
        },
        xtick=data,
        nodes near coords={
            \pgfmathprintnumber[precision=1]{\pgfplotspointmeta}
        }
    ]
    \addplot 
    	coordinates {
        	(GOG,58.6)
        	(Предл. метод,85.3)
        	(ResNet+TL,86.67)
        	(MobileNet+DML,87.73)
        };
    \addplot 
    	coordinates {
        	(MobileNet+DML,91.66)
        	(ResNet+TL,91.75)
        	(Предл. метод,92.91)
        };
    \legend{Запрос из одного изображения,Запрос из множества изображений}
    \end{axis}
    \end{tikzpicture}
    
    \hyperlink{etalon_collection}{\beamerbutton{Эталонные коллекции}}
    
\end{frame}

%-------------------------------------------------------------------------------
\begin{frame}[label=binary_precision]{Оценка бинарных дескрипторов}

    \begin{center}
    \begin{tikzpicture}
    \begin{axis}[
        axis on top,
        xlabel={Размер дескриптора, бит},
        ylabel={Точность rank-1, \%},
        xmin=0, xmax=600,
        ymin=48, ymax=87,
        xtick={0, 128,256,512},
        ytick={65,70,75,80,85},
        legend pos=south east,
        ymajorgrids=true,
        grid style=dashed,
        cycle list name=color,
        axis lines=left,
        axis y discontinuity=parallel,
        every axis plot/.append style={line width=2pt},
        % axis line style={opacity=0},
    ]
     
    \addplot
        coordinates {
        (128,75.2)(256,79.15)(512,83.4)
        };
    \addlegendentry{Предложенный алгоритм}
    
    \addplot
        coordinates {
        (128,71.79)(256,78.50)(512,83.9)
        };
    \addlegendentry{Алгоритм НБВ}

    \addplot
        coordinates {
        (128,68.34)(256,75.15)(512,78.50)
        };
    \addlegendentry{Выделение гл. компонент}
    
    \addplot
        coordinates {
        (128,65.17)(256,77.46)(512,83.72)
        };
    \addlegendentry{Случайный лес}
    
    \end{axis}
    \end{tikzpicture}
    \end{center}
    
    \hyperlink{binary_descriptor}{\beamerbutton{Методы бинаризации}}
    
\end{frame}

%-------------------------------------------------------------------------------
\begin{frame}{Оценка возможности переноса на \\ другие независимые данные}

    \begin{center}
    \begin{tikzpicture}
    \begin{axis}[
        ybar, axis on top,
        font=\footnotesize,
        height=0.7\textheight,width=\textwidth,
        bar width=0.7cm,
        ymajorgrids,
        major grid style={draw=white},
        enlarge y limits={value=.1,upper},
        ymin=0, ymax=100, ytick={0,20,40,60,80,100},
        axis x line*=bottom,
        y axis line style={opacity=0},
        tickwidth=0pt,
        enlarge x limits=true,
        legend style={
            at={(0.5,-0.2)},
            anchor=north,
            legend columns=-1,
            /tikz/every even column/.append style={column sep=0.5cm}
        },
        ylabel={rank-1, \%},
        symbolic x coords={
            Market/Market,
            Market/CUHK,
            CUHK/CUHK,
            CUHK/Market},
        xtick=data,
        nodes near coords={
            \pgfmathprintnumber[precision=1]{\pgfplotspointmeta}
        }
    ]
    \addplot 
    	coordinates {
        	(Market/Market,85.33)
        	(Market/CUHK,84.91)
        	(CUHK/CUHK,92.67)
        	(CUHK/Market,57.92)
        };
    \addplot 
    	coordinates {
    	    (Market/Market,83.72)
    	    (Market/CUHK,80.45) 
    		(CUHK/CUHK,89.94)
    		(CUHK/Market,48.30)
    	};
    \legend{Вещественный,Бинарный}
    \end{axis}
    \end{tikzpicture}
    \end{center}
    
\end{frame}

%-------------------------------------------------------------------------------
\begin{frame}{Программная реализация}

    \begin{itemize}
        \item Обучение и тестирование занимает 2 часа
        \item Построение одного дескриптора -- 1 мс
        \item Код доступен по ссылке: \href{https://github.com/vslutov/reidentification}{https://github.com/vslutov/reidentification}
    \end{itemize}
    
    \begin{center}
        \includegraphics[height=0.2\textheight]{image7.eps}
        \vspace{2pt}
        \includegraphics[height=0.2\textheight]{image8.eps}
        \vspace{2pt}
        \includegraphics[height=0.2\textheight]{image9.eps}
    \end{center}
    
\end{frame}

%-------------------------------------------------------------------------------
\begin{frame}{Результаты}
    \begin{itemize}
        \item С задачей поиска по нескольким изображениям алгоритм справился лучше аналогов по мере качества rank‑1 на тестовой коллекции Market1501
        \item Проведенная экспериментальная оценка показала, что алгоритм применим на других данных, полученных из независимого источника
        \item Бинарная модификация уступила базовому алгоритму по точности, но полученные дескрипторы занимают на 1--2 порядка меньше памяти
    \end{itemize}
\end{frame}

%-------------------------------------------------------------------------------
\section{Скрытые слайды}

%-------------------------------------------------------------------------------
\begin{frame}[label=binary_descriptor]{Методы бинаризации}

    \begin{itemize}
        \item Выделение значимых элементов методом случайного леса
        \item Уменьшение размерности методом главных компонент и бинаризация
        \item Нейросетевая бинаризация с помощью добавления специальных слоев -- полносвязной сигмоиды, алгоритма непосредственного бинарного встраивания
    \end{itemize}
    
    \hyperlink{binary_precision}{\beamerbutton{Назад}}
\end{frame}

%-------------------------------------------------------------------------------
\begin{frame}[label=etalon_collection]{Критерии и эталонная коллекция}

    \small
    
    \begin{tabular}{ p{2.5cm} | p{1.5cm} p{1.5cm} p{1.5cm} p{1.5cm} } 
        Коллекция данных & VIPeR & PRID & CUHK03 & Market1501 \\
        \hline
        Число личностей & 632 & 385 & 1467 & $\mathbf{1501}$ \\
        Примеров на человека & 1 & 1 & 2-10 & $\mathbf{\approx 15}$ \\
        Размер кадров & $128\times48$ & $128\times64$ & $160\times60$ & $128\times64$ \\
        Число камер & 2 & 2 & 2 & $\mathbf{6}$ \\
    \end{tabular}
    
    \bigskip

    В качестве эталонной коллекции выбрана наиболее полная -- Market1501
    
    \smallskip
    
    В качестве основной меры качества выбрана стандартная мера на этой коллекции rank-1: часть запросов, для которой первый ответ алгоритма ранжирования верный 
    
    \hyperlink{market1501_benchmark}{\beamerbutton{Назад}}
    
\end{frame}

%-------------------------------------------------------------------------------
\begin{frame}[label=hard_things]{Трудности}

    \begin{itemize}
        \item Низкое разрешение камер
        \item От камеры к камере меняется освещение, ракурс, окружение (возможны перекрытия)
    \end{itemize}
    
    \hyperlink{reason}{\beamerbutton{Назад}}
    
\end{frame}

\end{document}